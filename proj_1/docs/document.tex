\documentclass[a4paper,12pt]{article}

\newcommand*{\fg}[4][\textwidth]{
    \begin{figure}[!htb]
        \begin{center}
            \includegraphics[width=#1]{#2}
            \caption{#3}
            \label{rys:#4}
        \end{center}
    \end{figure}
}

%komenda do oznaczania elementów w tekście
% Parametr #1: Typ elementu (rys, lst)
% Parametr #2: Identyfikator elementu
\newcommand*{\Oznacz}[2]{
\ref{#1:#2} (s. \pageref{#1:#2})
}

%komenda do oznaczania zdjęć w tekście
% Parametr #1 (opcjonalny): Nazwa typu elementu (domyślnie Rysunek)
% Parametr #2: Identyfikator rysunku
\newcommand*{\OznaczZdjecie}[2][Rys.]{
#1 \Oznacz{rys}{#2}
}

%komenda do oznaczania kodu w tekście
% Parametr #1: Identyfikator kodu
\newcommand*{\OznaczKod}[1]{
\Oznacz{lst}{#1}
}

%komenda do wstawiania listingów kodu
% Parametr #1: Opis listingu wyświetlany pod listingiem
% Parametr #2: Identyfikator listingu
% Parametr #3: Ścieżka do pliku z kodem
\newcommand*{\ListingFile}[3]{
    \lstinputlisting[caption=#1, label={lst:#2}, language=C++]{#3}
}

% Packages
\usepackage[utf8]{inputenc}
\usepackage[T1]{fontenc}
\usepackage{lmodern}
\usepackage[polish]{babel}
\usepackage{amsmath}
\usepackage{amsfonts}
\usepackage{amssymb}
\usepackage{graphicx}
\usepackage{hyperref}
\usepackage{geometry}
\usepackage{caption}
\usepackage{listings}
\usepackage{xcolor}
\geometry{margin=0.5in}

\lstdefinestyle{csharpstyle}{
    language=[Sharp]C,
    basicstyle=\ttfamily\footnotesize,
    keywordstyle=\color{blue!90!black}\bfseries,
    commentstyle=\color{gray}\itshape,
    stringstyle=\color{green!70!black},
    numbers=left,
    numberstyle=\ttfamily\tiny\color{gray}\raisebox{0.5ex},
    stepnumber=1,
    numbersep=5pt,
    breaklines=true,
    frame=single,
    backgroundcolor=\color{gray!5},
    showspaces=false,
    showstringspaces=false,
    showtabs=false,
    tabsize=4,
    captionpos=b,
    linewidth=\linewidth,
    xleftmargin=20pt,
    xrightmargin=5pt,
    framexleftmargin=15pt,
    framexrightmargin=5pt,
    rulecolor=\color{blue!30},
    literate=  % Polish characters preserved
      {ą}{{\k{a}}}1 
      {ć}{{\'c}}1 
      {ę}{{\k{e}}}1 
      {ł}{{\l}}1 
      {ń}{{\'n}}1
      {ó}{{\'o}}1 
      {ś}{{\'s}}1 
      {ź}{{\'z}}1 
      {ż}{{\.{z}}}1
      {Ą}{{\k{A}}}1 
      {Ć}{{\'C}}1 
      {Ę}{{\k{E}}}1 
      {Ł}{{\L}}1
      {Ń}{{\'N}}1 
      {Ó}{{\'O}}1 
      {Ś}{{\'S}}1 
      {Ź}{{\'Z}}1 
      {Ż}{{\.{Z}}}1,
    emph={Console, WriteLine, ReadLine, ToString, int, string, double, float, bool, class, interface, 
          public, private, protected, static, void, new, get, set, using, namespace, return, if, else, 
          for, foreach, while, do, try, catch, finally, switch, case, break, continue, this, base, 
          var, List, Dictionary, Task, async, await},
    emphstyle=\color{magenta!70!black},
    lineskip=2pt,
    aboveskip=10pt,
    belowskip=10pt,
    numberblanklines=false
}

\captionsetup[lstlisting]{justification=centering}

% \captionsetup{labelformat=empty} % Usunięcie numeracji tabeli

% Define an invisible box for phantom text
\newcommand{\invisiblephantom}{\phantom{(X)}}

\begin{document}
% Please add the following required packages to your document preamble:
% \usepackage{graphicx}
% Please add the following required packages to your document preamble:
% \usepackage{graphicx}
\begin{table}[]
\centering
\resizebox{\textwidth}{!}{%
\begin{tabular}{|lll|}
\hline
\multicolumn{3}{|c|}{\begin{tabular}[c]{@{}c@{}}Wydział Nauk Inżynieryjnych ANS w Nowym Sączu \\ Optymalizacja dyskretna - projekt\end{tabular}}                                                                                                              \\ \hline
\multicolumn{2}{|l|}{\begin{tabular}[c]{@{}l@{}}Temat: Programowanie dyskretne. Wprowadzenie do programowania w języku c\#.\\ Operacje na tablicach, funkcje.\end{tabular}}                     & \begin{tabular}[c]{@{}l@{}}Symbol:\\ OD\_P1\end{tabular}    \\ \hline
\multicolumn{1}{|l|}{\begin{tabular}[c]{@{}l@{}}Nazwisko i imię:\\ \\ Ryczek Arkadiusz\end{tabular}} & \multicolumn{1}{l|}{\begin{tabular}[c]{@{}l@{}}Ocena sprawozdania:\\  \\ \invisiblephantom\end{tabular}} & \begin{tabular}[c]{@{}l@{}}Zaliczenie:\\ \\  \invisiblephantom\end{tabular} \\ \hline
\multicolumn{1}{|l|}{\begin{tabular}[c]{@{}l@{}}Data wykonania\\ ćwiczenia:\end{tabular}}            & \multicolumn{2}{l|}{\begin{tabular}[c]{@{}l@{}}Oceniane efekty uczenia się:\\ EUU1=……………… ,EUU2=……………… ,EUU3=…………………, EUK1=………………….\end{tabular}}      \\ \hline
\end{tabular}%
}
\end{table}
\section*{Wstęp}
Celem zajęć było zapoznanie się z podstawowymi elementami języka C\# oraz środowiskiem programistycznym Visual Studio Code (przygotowanie IDE do pracy z C\#, pobranie odpowiednich rozszerzeń, .NET itd.). W ramach ćwiczeń wykonano kilka prostych programów, które miały na celu utrwalenie poznanych konstrukcji językowych, a konkretnie operacji na tablicach oraz zapisu/odczytu danych z plików tekstowych. Poniżej przedstawiono opis wykonanych zadań oraz kod źródłowy poszczególnych programów.
\section*{Zadanie 1}
\begin{quote}
Napisz program w języku c\#, który wczytuje elementy numeryczne do tablicy dwuwymiarowej nxn. Wartości można podać z klawiatury oraz wczytać liczby losowe z zakresu od 0-100. Program posiada menu wyboru:
\begin{enumerate}
    \item Wczytaj klawiatura,
    \item Wczytaj losowe,
    \item Wyświetl tablice,
    \item ESC - wyjście.
\end{enumerate}
\end{quote}
Poniżej przedstawiono kod źródłowy programu realizującego powyższe zadanie.
\begin{lstlisting}[style=csharpstyle, caption={Kod źródłowy programu realizującego zadanie 1}, label={lst:zad1}]
#nullable disable

Console.WriteLine("Enter the size of the matrix:");
int size = int.Parse(Console.ReadLine());
int[,] matrix = new int[size, size];
Random rand = new Random();

int option;

for (;;)
{
    Console.WriteLine(
        "1. Load matrix with your own values\n2. Load matrix with random values\n3. Display the matrix\n4. Exit"
    );
    option = int.Parse(Console.ReadLine());

    switch (option)
    {
        case 1:
            Console.WriteLine("Input your own values...");
            for (int i = 0; i < size; i++)
            {
                for (int j = 0; j < size; j++)
                {
                    Console.Write($"Enter value for position [{i},{j}]: ");
                    matrix[i, j] = int.Parse(Console.ReadLine());
                }
            }
            break;
        case 2:
            Console.WriteLine("Loading random values.");
            for (int i = 0; i < size; i++)
            {
                for (int j = 0; j < size; j++)
                {
                    matrix[i, j] = rand.Next(1, 101);
                }
            }
            break;

        case 3:
            Console.WriteLine("Displaying the matrix:");
            for (int i = 0; i < size; i++)
            {
                for (int j = 0; j < size; j++)
                {
                    Console.Write(matrix[i, j] + "\t");
                }
                Console.WriteLine();
            }
            break;
        case 4:
            Console.WriteLine("Exiting...");
            return;
        default:
            Console.WriteLine("Invalid option.");
            break;
    }
}
\end{lstlisting}

Program implementuje menu wyboru umożliwiające użytkownikowi wczytanie danych do tablicy dwuwymiarowej. Funkcjonalności programu obejmują:

\begin{enumerate}
    \item Wprowadzenie wartości z klawiatury - użytkownik może ręcznie wprowadzić wartości dla każdej pozycji w tablicy,
    \item Wygenerowanie liczb losowych z zakresu 1-100 - program automatycznie wypełnia tablicę losowymi wartościami,
    \item Wyświetlenie zawartości tablicy - program prezentuje aktualną zawartość tablicy w formie czytelnej dla użytkownika,
    \item Wyjście z programu - zakończenie działania aplikacji.
\end{enumerate}

Program działa w pętli nieskończonej, umożliwiając wielokrotne wykonywanie operacji do momentu wyboru opcji wyjścia.

\section*{Zadanie 2}
\begin{quote}
Napisz funkcję do odczytu i zapisu danych z pliku „p1.txt”. Wklej do sprawozdania zrzut ekranu z działającym
programem.
Menu:
\begin{enumerate}
    \item Wczytaj klawiatura,
    \item Wczytaj losowe,
    \item Wyświetl tablice,
    \item Wczytaj z pliku,
    \item Zapisz do pliku,
    \item ESC wyjście.
\end{enumerate}
\end{quote}

Poniżej przedstawiono kod źródłowy programu realizującego powyższe zadanie.
\begin{lstlisting}[style=csharpstyle, caption={Kod źródłowy programu realizującego zadanie 2}, label={lst:zad2}]
#nullable disable
using System.IO;

Console.WriteLine("Enter the size of the matrix:");
int size = int.Parse(Console.ReadLine());
int[,] matrix = new int[size, size];
Random rand = new Random();

int option;

for (; ; )
{
    Console.WriteLine(
        "1. Load matrix with your own values\n2. Load matrix with random values\n3. Display the matrix\n4. Load matrix from file\n5. Save matrix to file\n6. Exit"
    );
    option = int.Parse(Console.ReadLine());

    switch (option)
    {
        case 1:
            Console.WriteLine("Input your own values...");
            for (int i = 0; i < size; i++)
            {
                for (int j = 0; j < size; j++)
                {
                    Console.Write($"Enter value for position [{i},{j}]: ");
                    matrix[i, j] = int.Parse(Console.ReadLine());
                }
            }
            break;
        case 2:
            Console.WriteLine("Loading random values.");
            for (int i = 0; i < size; i++)
            {
                for (int j = 0; j < size; j++)
                {
                    matrix[i, j] = rand.Next(1, 101);
                }
            }
            break;

        case 3:
            Console.WriteLine("Displaying the matrix:");
            for (int i = 0; i < size; i++)
            {
                for (int j = 0; j < size; j++)
                {
                    Console.Write(matrix[i, j] + "\t");
                }
                Console.WriteLine();
            }
            break;
        case 4:
            try
            {
                Console.Write("Enter filename (e.g., p1.txt): ");
                string filename = Console.ReadLine();
                StreamReader sr = new StreamReader(filename);

                int fileSize = int.Parse(sr.ReadLine());

                if (fileSize != size)
                {
                    size = fileSize;
                    matrix = new int[size, size];
                }

                for (int i = 0; i < size; i++)
                {
                    string[] values = sr.ReadLine().Split(' ');
                    for (int j = 0; j < size; j++)
                    {
                        matrix[i, j] = int.Parse(values[j]);
                    }
                }

                sr.Close();
                Console.WriteLine("Matrix loaded successfully from file.");
            }
            catch (Exception e)
            {
                Console.WriteLine("Exception: " + e.Message);
            }
            break;
        case 5:
            try
            {
                StreamWriter sw = new StreamWriter("p1.txt");

                sw.WriteLine(size);

                for (int i = 0; i < size; i++)
                {
                    for (int j = 0; j < size; j++)
                    {
                        sw.Write(matrix[i, j]);
                        if (j < size - 1)
                            sw.Write(" ");
                    }
                    sw.WriteLine();
                }

                sw.Close();
                Console.WriteLine("Matrix saved successfully to p1.txt");
            }
            catch (Exception e)
            {
                Console.WriteLine("Exception: " + e.Message);
            }
            break;

        case 6:
            Console.WriteLine("Exiting...");
            return;
        default:
            Console.WriteLine("Invalid option.");
            break;
    }
}
\end{lstlisting}

Program rozszerza funkcjonalność z zadania pierwszego o operacje na plikach. Dodano dwie nowe opcje menu:

\begin{enumerate}
    \item[4.] Wczytanie tablicy z pliku - program odczytuje rozmiar tablicy oraz jej zawartość z pliku tekstowego. Format pliku zawiera w pierwszej linii rozmiar tablicy, a w kolejnych liniach wartości elementów oddzielone spacjami,
    \item[5.] Zapis tablicy do pliku - program zapisuje aktualną zawartość tablicy do pliku „p1.txt" w formacie zgodnym z opcją odczytu.
\end{enumerate}

Operacje na plikach są zabezpieczone mechanizmem obsługi wyjątków (try-catch), co zapewnia stabilność działania programu w przypadku problemów z dostępem do pliku lub nieprawidłowym formatem danych.

\section*{Podsumowanie}
W ramach ćwiczeń zapoznano się z podstawami programowania w języku C\#. Wykonane zadania pozwoliły na praktyczne zastosowanie poznanych konstrukcji językowych, takich jak operacje na tablicach oraz obsługa plików tekstowych, co będzie przydatne na dalszych zajęciach z optymalizacji dyskretnej.

\end{document}